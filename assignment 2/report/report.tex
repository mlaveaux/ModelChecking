\documentclass[10pt,a4paper]{article}

\usepackage[utf8]{inputenc}
\usepackage{amsmath}
\usepackage{amsfonts}
\usepackage{amssymb}
\usepackage{tikz}
\usepackage{pgf}
\usepackage{pgfplots}
\usepackage{hyperref}
\usepackage{comment}
\usepackage{algorithm}
\usepackage[noend]{algpseudocode}
\usepackage{parskip}
\usepackage{listings}
\usepackage{xcolor}

\pgfplotsset{compat=1.3}
\title{Implementation and evaluation of the small progress measures algorithm}

\author{Olav Bunte (0803961, o.bunte@student.tue.nl),\\
Maurice Laveaux (0813568, m.laveaux@student.tue.nl),\\
Ziad Ben Snaiba (0748095, z.b.snaiba@student.tue.nl)}

\date{\today}

\begin{document}
\maketitle

\section{Introduction}
This report elaborates on our work and findings of assignment two. The goal of this assignment is to implement and evaluate the small progress measures algorithm as described in \cite{spmpaper}. First we will address the design and implementation of the algorithm in section \ref{design}, along with alternative lifting strategies to make the algorithm more efficient. Then in section \ref{eval} we will show the performance of each lifting strategy using self-made and provided parity games. Lastly, we will reach a conclusion in section \ref{conc}.

\section{Design and implementation}\label{design}

\subsection{Parity Game}

\subsection{Small progress measures}

\subsection{Lifting orders}

\subsubsection{In degree}

\subsubsection{order2}


\section{Evaluation}\label{eval}

\subsection{Self-made parity games}

\subsection{Dining Philosophers}

\subsection{Elevator}


\section{Conclusion}\label{conc}


\begin{thebibliography}{9}
\bibitem{spmpaper} M. Jurdzi\'{n}ski: Small Progress Measures for Solving Parity Games, March 2000
\end{thebibliography}


\newpage
\appendix









\end{document}